\chapter{Applications}
\begin{definition}[Application]
	Soient E et F deux ensembles. $f:E \to F$ est une application si pour chaque $x \in E$, on associe un élément de F noté $f(x)$
\end{definition}

\begin{definition}[Injectivité]
	Soit $f:E \to F$, on dit que $f$ est injective si pour chaque élément de F, il y a au plus un élément de E qui y est associé. Autrement dit :
	\begin{align*}
		\text{f injective} \iff \{\forall (x_1, x_2) \in E^2, f(x_1) = f(x_2) \implies x_1 = x_2\}
	\end{align*}
\end{definition}

\begin{definition}[Surjectivité]
	Soit $f:E \to F$, on dit que $f$ est surjective si pour chaque élément de F, il y a au moins un élément de E qui y est associé.
	Autrement dit :
	\begin{align*}
		\text{f surjective} \iff \{\forall y \in F, \exists x \in E, y = f(x)\}
	\end{align*}
\end{definition}

\begin{definition}[Bijectivité]
	Soit $f:E \to F$, on dit que $f$ est bijective si elle est injective et surjective, c'est-à-dire que pour chaque élément de F, il y a exactement un élément de E qui y est associé.
	Autrement dit :
	\begin{align*}
		\text{f bijective} \iff \{\forall y \in F, \exists! x \in E, y = f(x)\}
	\end{align*}
\end{definition}

\begin{definition}[Ensemble fini]
	$
		\text{Un ensemble E est un ensemble fini non-vide} \iff \exists n \in \N^*, \exists \text{une application bijective de } \{1,2,\ldots,n\} \text{ dans } E
	$
\end{definition}

\begin{definition}[Fonction réciproque]
	Soient E et F deux ensembles. Supposons que $f:E \to F$ est une application bijective. On peut définir l'application 
	\begin{align*}
		f^{-1} : 
		\begin{cases}
			F &\to E \\
			y &\mapsto x
		\end{cases}
	\end{align*}
	comme étant la réciproque de $f$.
\end{definition}

\begin{definition}[Composition]
	Soient $f$ et $g$ deux applications telles que :
	$f:E \to F$ et $g:F \to G$ on a l'application $g \circ f : E \to G$ qui est définie comme étant la composée de $f$ et de $g$.
\end{definition}

\begin{definition}[Image directe et image réciproque]
	Soient $f:E \to F$ une application, A une partie de E et B une partie de F. Nous avons :
	\begin{align*}
		f(A) = \{f(x), x \in A\} &\text{ : image directe} \\
		f^{-1}(B) = \{x \in E, f(x) = B\} &\text{ : image réciproque}
	\end{align*}
\end{definition}
\clearpage
\begin{proposition}[Propriétés sur les images directes et réciproques]
	Soient $f:E \to F$ une application et A, B des parties de F.
	\begin{enumerate}
		\item \label{prop_img_1} $f^{-1}(F \backslash A) = E \backslash f^{-1}(A)$
		\item \label{prop_img_2} $f^{-1}(A \cup B) = f^{-1}(A) \cup f^{-1}(B)$
		\item \label{prop_img_3} $f^{-1}(A \cap B) = f^{-1}(A) \cap f^{-1}(B)$
		\item \label{prop_img_4} $f(A \cup B) = f(A) \cup f(B)$
		\item \label{prop_img_5} $f(A \cap B) \subset f(A) \cap f(B)$
	\end{enumerate}
\end{proposition}

\begin{proof}
	\ref{prop_img_5} : $f(A \cap B) \subset f(A) \cap f(B)$ \\
	\par \noindent Soit $y \in f(A \cap B) = \{f(x), x \in A \cap B\}$, par définition : $\exists x \in A \cap B, y = f(x)$
	\begin{align*}
		x \in A \cap B &\iff x \in A \wedge x \in B \\
		x \in A &\implies y = f(x) \subset f(A) \\
		x \in B &\implies y = f(x) \subset f(B) 
	\end{align*}
	d'où $y \in f(A) \cap f(B)$
\end{proof} 

\begin{remarque}
	\begin{align*}
		f(A \cap B) \neq f(A) \cap f(B)
	\end{align*}
\end{remarque}