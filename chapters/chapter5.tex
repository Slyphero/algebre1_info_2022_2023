\chapter{Arithmétique}
\section{Divisibilité}
\begin{graybox}
\begin{definition}
Soient $a \in \Z, b \in \Z^*$. 
On dit que :
\begin{itemize}
\item $a$ est un multiple de $b \iff \exists q \in \Z, a = bq$
\item $b$ est un diviseur de $a \iff \exists q \in \Z, a = bq$
\item $b$ divise $a$ $\iff b \mid a$
\end{itemize}
\end{definition}
\end{graybox}

\begin{graybox}
\begin{theoreme}[Division euclidienne]
Soit $(a, b) \in \Z \times \Z^*$. Alors
\begin{align*}
\exists ! (q, r) \in \Z^2, a = bq + r,\ (0 \leq r < |b|)
\end{align*}
\textbf{Vocabulaire :}
\begin{itemize}
\item $a$ est appelé le \textit{dividende}
\item $b$ est appelé le \textit{diviseur}
\item $q$ est appelé le \textit{quotient}
\item $r$ est appelé le \textit{reste}
\end{itemize}
\end{theoreme}
\end{graybox}

\begin{proof}\cite{livre_prepa}
Nous devons montrer deux choses, \textit{l'existence} et \textit{l'unicité} du couple $(q, r)$
\begin{enumerate}
\item \textbf{Existence} 
\\
Montrons que $(q, r)$ existe.
\\
Supposons $a \in \N$ et considérons $M = \left\{ n \in \N \mid nb \leq a \right\}$ l'ensemble des multiples de $b$ inférieurs à $a$. $M$ est une partie de $\N$.
Nous avons deux propriétés :
\begin{enumerate}
\item $M$ est non vide car $0$ est un multiple de $b$ inférieur à $a$
\item $M$ est majoré par $a$ d'après sa définition.
\end{enumerate}
Ainsi, $M$ admet un plus grand élément que l'on notera $q$, vérifiant :
\begin{enumerate}
\item $qb \leq a$ car $q \in M$
\item $(q+1) b > a$ car $q + 1 > q$ sachant que $q$ est le plus grand élément de $M$, $q + 1 \notin M$.
\end{enumerate}
Posons $r = a - bq \Longleftrightarrow a = bq + r$.
Sachant que $a \geq bq$, $r \geq 0$. 
\\
On a $r < b$ car $b = (q+1)b - qb > a - bq = r$.
\\
Supposons que $a \in \Z$. Si $a$ est positif, on se ramène au cas précédent.
\\
Dans le cas où $a < 0$, $-a \geq 0$, ainsi, $\exists (q', r') \in \Z^2$ tel que 
\begin{align*}
-a &= bq' + r' \text{ avec } 0 \leq r' < |b| \\
\iff a &= b(-q') -r'
\end{align*}
Si $r' = 0$, on pose $q = -q'$ et $r = 0$ et on obtient le couple recherché. 
\\
Si $r' \neq 0$, $r' \in \llbracket 1, b - 1 \rrbracket$  et $a = b(-q' - 1) + (b - r')$, on pose $q = -q' - 1$ et $r = b - r'$ et on obtient le couple recherché.
\item \textbf{Unicité} \\
Pour cette partie, il suffit de supposer deux couples $(q, r) \in \Z^2$ et $(q', r') \in \Z^2$ et de montrer que $q = q'$ et $r = r'$.
\\
Commençons par $a = bq + r,\ (0 \leq r < |b|)$ et $a = bq' + r',\ (0 \leq r' < |b|)$. Comme $0 \leq r < b$ et $0 \leq r' < b$, on a : 
\begin{align*}
b|q' - q| = |r' - r| < b
\end{align*}
ce qui n'est possible que si $|q' - q| = 0$ ce qui implique que $q = q'$. Ceci entraîne donc $r = r'$ et donc on a montré que $(q, r) = (q', r')$
\end{enumerate}
\end{proof}

\section{PGCD et PPCM}
\begin{graybox}
\begin{definition}[PPCM et PGCD]
Soit $(a, b) \in \Z^2 \backslash \{(0, 0)\}$ tel que $ab \neq 0$
\begin{align*}
\mathcal{M} \colon = \left\{ m \in \Z \ | \ a \mid m \text{ et } b \mid m \right\} \Rightarrow \mathcal{M} \neq \phi \text{ car } ab \in \mathcal{M}
\end{align*}
\begin{align*}
\mathcal{M} \cap \N^* \leftarrow \text{ il y a le plus petit commun multiple (PPCM) }
\end{align*}
\begin{align*}
\mathcal{D} = \left\{ d \in \Z \mid d \mid a \text{ et } d \mid b \right\} \Rightarrow \mathcal{D} \neq \emptyset \text{ car } 1 \in \mathcal{D}
\end{align*}
On a : $d \mid a, b \implies |d| \leq m \min (|a|, |b|)$ et $\mathrm{Card}(\mathcal{D}) < \infty$
Il y a le plus grand élément $\leftarrow$ le plus grand commun diviseur (PGCD)
\end{definition}
\end{graybox}

\begin{graybox}
\begin{theoreme}[PPCM]
Soit $(a, b) \in \Z^* \times \Z^*$ et $m \in \Z,\ a \mid m \text{ et } b \mid m$. Alors $\mathrm{ppcm}(a, b) \mid m$
\end{theoreme}
\end{graybox}

\begin{proof}
Posons $\ell = \mathrm{ppcm}(a, b)$
\begin{align*}
\exists ! (q, r) \in \Z^2, \ m &= q\ell + r,\ 0 \leq r < \ell \\
\iff r &= m - q\ell,\ m \text{ et } \ell \text{ sont multiples de } a \text{ et } 
\\
&r \text{ est aussi un multiple de } a \text{ et } b
\end{align*}
Par la minimalité de $\ell$, $r = 0 \implies m = q \ell$
\end{proof}

\begin{graybox}
\begin{theoreme}[PGCD]
Soit $(a, b) \in \Z^* \times \Z^*$ et $d \in \Z,\ d \mid a \text{ et } d \mid b$. Alors $d \mid \mathrm{pgcd}(a, b)$
\end{theoreme}
\end{graybox}

\begin{proof}
Posons $m = \mathrm{pgcd}(a, b)$. Il suffit de montrer que 
\begin{align*}
\mathrm{pgcd}(m, d) = m
\end{align*}
Soit $\ell = \mathrm{ppcm}(m ,d)$, $\ell \geq m$, $a$ et $b$ sont multiples de $m$ et $d$
\\
D'après le théorème précédent : 
\begin{align*}
\ell \mid a \text{ et } \ell \mid b, \ l \leq m
\end{align*}
Sachant qu'on a $\ell \geq m$ et $\ell \leq m$, on en conclut que $\ell = m$
\end{proof}

\begin{graybox}
\begin{theoreme}
Soit $(a, b) \in (\N^*)^2 \implies ab = \mathrm{pgcd}(a, b)\mathrm{ppcm}(a, b)$
\end{theoreme}
\end{graybox}

\begin{graybox}
\begin{definition}[Nombres premiers entre eux]
	Soit $(a, b) \in (\Z^*)^2$
	\begin{align*}
	a \text{ et } b \text{ premiers entre eux } \iff \mathrm{pgcd}(a, b) = 1
	\end{align*}
\end{definition}
\end{graybox}

\section{Algorithme d'Euclide}
\begin{graybox}
\begin{proposition}[Algorithme d'Euclide]
Soient $a \in \Z^*$, $b \in \Z^*$ tel que 
\begin{align*}
|a| > |b| \implies \exists ! (q, r) \in \Z^2, a = b q + r, 0 \leq r < |b| \\
\mathrm{pgcd}(a, b) = \mathrm{pgcd}(b, a) = \mathrm{pgcd}(b, a - qb) = \mathrm{pgcd}(b, r) \\
\end{align*}
\begin{equation*}
\mathrm{pgcd}(a, b) = \mathrm{pgcd}(b, r)
\end{equation*}
Si $r = 0 \implies a = q b,\ \mathrm{pgcd}(a, b) = b$ 
Supposons que $r = 0$ :
\begin{equation*}
\exists ! (q_1, r_1),\ b = q_1 r + r_1, \ 0 \leq r_1 < r
\end{equation*}
Si $r_1 \neq 0 \implies \exists ! (q_2, r_2),\ r = q_2 r_1 + r_2,\ 0 \leq r_2 < r_1$ \\
$\vdots$ \\
Si $r_{n - 2} \neq 0 \implies \exists ! (q_{n - 1}, r_{n - 1}),\ r_{n - 3} = q_{n-1}r_{n-2} + r_{n - 1},\ 0 \leq r_{n-1} < r_{n-2}$

\begin{equation*}
\exists q_n,\ r_{n-2} = q_n r_{n-1}
\end{equation*}
\begin{align*}
\mathrm{pgcd}(a, b) &= \mathrm{pgcd}(b, r) \\
				   &= \mathrm{pgcd}(r, r_1) \\
				   &= \mathrm{pgcd}(r_1, r_2) \\
				   &\vdots \\
				   &= \mathrm{pgcd}(r_{n-2}, r_{n-1}) \\
				   &= \mathrm{pgcd}(q_n r_{n-1}, r_{n-1}) = r_{n-1}
\end{align*}
\end{proposition}
\end{graybox}

\begin{exemple}~ 
\begin{enumerate}
\item $\mathrm{pgcd}(72, 58)$
\begin{align*}
72 &= 58 \times 1 + 14 \\
58 &= 14 \times 4 + 2 \\
14 &= 2 \times 7 + 0
\end{align*}
On en conclut que $\mathrm{pgcd}(72, 58) = 2$

\item $\mathrm{pgcd}(625, 216)$
\begin{align*}
625 &= 216 \times 2 + 193 \\
216 &= 193 \times 1 + 23 \\
193 &= 23 \times 8 + 9 \\
23 &= 9 \times 2 + 5 \\
9 &= 5 \times 1 + 4 \\
5 &= 4 \times 1 + 1
\end{align*}
On en conclut que $\mathrm{pgcd}(625, 216) = 1$
\end{enumerate}
\end{exemple}

\begin{graybox}
\begin{theoreme}[Identité de Bézout]
Soient $(a, b) \in \Z^2$ 
\begin{align*}
\exists (u, v) \in Z^2 \text{ tel que } au + bv = \mathrm{pgcd(a, b)}
\end{align*}
\end{theoreme}
\end{graybox}

\begin{graybox}
\begin{corollaire}
Soient $(a, b) \in (\Z^*)^2$, $d \in \Z$
\begin{align*}
\exists (u, v) \in \Z^2 \text{ tel que } au + bv = d \iff \mathrm{pgcd}(a,b) \mid d
\end{align*}
\end{corollaire}
\end{graybox}

Trouver les $(x, y) \in \Z$ tels que $ax + by = d$ et $\mathrm{pgcd}(a, b) \mid d$
\\
\textbf{Théorème de Bézout} : $\exists (x_0, y_0) \in \Z^2$ tel que $ax_0 + by_0 = d$
\begin{align*}
\begin{cases}
&ax + by = d \\
&ax_0 + by_0 = d
\end{cases}
\implies 
a(x - x_0) + b(y - y_0) = 0 \iff a(x - x_0) = b(y_0 - y) \text{ multiple  }k \text{ ppcm}(a,b)
\end{align*}
\begin{align*}
\exists k \in \Z \text{ tq } a(x - x_0) = b(y - y_0) = \mathrm{ppcm}(a, b)k \\
\begin{cases}
&x = x_0 + \frac{\mathrm{ppcm}(a, b)k}{a} \\
&y = y_0 - \frac{\mathrm{ppcm}(a,b)k}{b}
\end{cases}
\end{align*}
d'où 
\begin{align*}
\left\{ (x,y) \in \Z^2 \mid ax + by = d \right\} = (x_0, y_0) + \Z \left( \frac{\mathrm{ppcm}(a, b)}{a}, \frac{\mathrm{ppcm}(a, b}{b}\right)
\end{align*}

\begin{exemple}
$a = 75$ et $b = 42$ \\
\begin{align*}
&75 = 42 \cdot 1 + 33 \\
&42 = 33 \cdot 1 + 9 \\
&33 = 9 \cdot 3 + 6 \\
&9 = 6 \cdot 1 + 3\\
&6 = 3 \cdot 2 + 0
\end{align*}
On remonte dans l'algorithme d'Euclide
\begin{align*}
3 &= 9 - 6  \\
3 &= (42 - 33) - (33 - 9 \cdot 3) \\
3 &= (42 - (75 - 42)) - ((75 - 42) - (42 - 33) 3)  \\
3 &= (42 - (75 - 42)) - ((75 - 42) - (42 - (75 - 42))3) \\
3 &= 75 \cdot (-5) + 42 \cdot 9 
\end{align*}
\end{exemple}

\begin{graybox}
\begin{lemme}[Lemme de Gauss]
$(a, b) \in \Z^*$ tels que $a$ et $b$ sont premiers entre eux (leur pgcd est 1)
\begin{align*}
c \in \Z \text{ tq } a \mid bc \implies a \mid c
\end{align*}
\end{lemme}
\end{graybox}

\begin{proof}
\begin{align*}
\mathrm{pgcd}(a, b) = 1 &\implies \exists (u, v) \in \Z^2 \text{ tq } au + bv = 1 \\
&\implies a(cu) + b(cv) = c \\
&\implies \mathrm{pgcd}(a,bc) \mid c
\end{align*} 
\end{proof}

\section{Nombres premiers}
\begin{graybox}
\begin{definition}[Nombres premiers]
$p \in \N^*$ est dit premier si 
\begin{align*}
\exists d \in \N^* \text{ tq } d \mid p \implies d \in \left\{ 1, p \right\}
\end{align*}
\end{definition}
\end{graybox}

\begin{exemple}
$2, 3, 5, 7, 11, 13, 17, 19, 23, 29, 31, 37$ sont des nombres premiers
\end{exemple}

\begin{remarque}
$F_n = 2^{2^n} + 1$ est une suite composée exclusivement de nombres premiers.
\end{remarque}

\begin{graybox}
\begin{theoreme}[Théorème d'Euclide]
	Il existe une infinité de nombres premiers.
\end{theoreme}
\end{graybox}

\begin{proof}
Supposons qu'il existe $k$ nombres premiers $p_1, p_2, \cdots, p_k$ 
\begin{align*}
N \colon = p_1 p_2 \cdots p_k + 1 \implies p_i \nmid N
\end{align*}
\end{proof}

\begin{graybox}
\begin{lemme}
Soit $n \in \N$ tq $n \geq 2$. \\ 
Soit $p$ le plus petit diviseur de $n$ tq $p > 2 \implies p$ premier  
\end{lemme}
\end{graybox}

\begin{proof}
Si $p$ n'était pas premier : $1 < \exists d < p$ tq $d \mid p$
\\
$d \mid p$ et $p \mid n \implies d \mid n$ ce qui reviendrait à contredire la minimialité de $p$
\end{proof}

\begin{graybox}
\begin{theoreme}[Décomposition en facteurs premiers]
Soit $n \in \N^*$ tq $n \geq 2$. 
Il existe une unique écriture de $n$ sous la forme de :
\begin{align*}
p_1^{\alpha_1} p_2^{\alpha_2}  \cdots p_k^{\alpha_k}
\end{align*}
\begin{enumerate}
\item $p_i$ premiers
\item $\alpha_i \in \N^*$
\item $p_1 < p_2 < \cdots < p_k$
\end{enumerate}
\end{theoreme}
\end{graybox}

\begin{proof}
\textbf{Existence} : On procède par récurrence forte en utilisant le Lemme de Gauss \\
\textbf{Unicité} : On utilise le Lemme de Gauss
\end{proof}

\begin{graybox}
\begin{proposition}[PGCD à partir de la décomposition en facteurs premiers]
Soient $(a, b) \in \Z^2$, pour déterminer leur PGCD, on peut se servir de leurs décomposition en facteurs premiers.
\begin{align*}
a &= p_1^{\alpha_1} p_2^{\alpha_2} \cdots p_k^{\alpha_k} (k \in \N)\\
b &= n_1^{\beta_1} n_2^{\beta_2} \cdots p_i^{\beta_i} (i \in \N)
\end{align*}
pgcd$(a, b)$ correspond aux produits des facteurs premiers communs.
\end{proposition}
\end{graybox}

\begin{remarque}[Décomposer en facteurs premiers]
Pour décomposer facilement un nombre en facteurs premiers. \begin{enumerate}
\item On divise par le diviseur premier le plus faible tel que le reste soit nul
\item On refait de même avec le quotient, il faut que le diviseur premier soit supérieur ou égal au précédent.
\item On continue jusqu'à finir avec un quotient premier
\end{enumerate}
\end{remarque}

\begin{exemple}
Décomposons $423$ en facteurs premiers.
\begin{align*}
\frac{423}{3} &= 141 \\
\frac{141}{3} &= 47 \\
\end{align*}
Sachant que $47$ est premier, on obtient la décomposition
\begin{align*}
423 = 3 \times 3 \times 47 = 3^2 \times 47
\end{align*}
\end{exemple}

\begin{exemple}
Calculons le PGCD de 624 et 408.\\
On a :
\begin{align*}
624 &= 2^4 \times 3 \times 13 \\
408 &= 2^3 \times 3 \times 17
\end{align*}
On remarque que $2^3$ et $3$ sont communs aux deux décompositions.
\begin{align*}
\mathrm{pgcd}(624, 408) = 2^3 \times 3 = 24
\end{align*}
\end{exemple}

\section{Congruences}
\begin{graybox}
\begin{definition}[Congruence]
	Soient $(a, b) \in \Z^2$ et $n \in \N, n \geq 2$ \\
	On dit que $a$ et $b$ sont congrus modulo $n$ s'il existe un $k \in \Z$ tel que :
	\begin{align*}
	n \mid a - b \iff a - b = kn
	\end{align*}
	On note : 
	\begin{align*}
	a \equiv b \ [n] \iff a = b \ (\mathrm{mod}\ n)
	\end{align*}
\end{definition}
\end{graybox}

\begin{exemple}
\begin{align*}
9 \equiv 2 [7] \iff 9 \equiv 9 [7]
\end{align*}
\end{exemple}

\begin{graybox}
\begin{proposition}
Pour $(a, b, c) \in \Z^3$ et $n \in \N, n \geq 2$ 
\begin{enumerate}
\item $a \equiv a \ [n]$ (Réflexivité)
\item $a \equiv b \ [n] \implies b \equiv a \ [n]$ (Symétrie)
\item $a \equiv b \ [n],\ b \equiv c \ [n] \implies a \equiv c \ [n]$ (Transitivité)
\item $a \equiv b \ [n],\ c \equiv d \ [n] \implies a + c \equiv b + d \ [n]$
\item $a \equiv b \ [n],\ c \equiv d \ [n] \implies ac \equiv bc \ [n]$
\item $a \equiv b \ [n] \implies a^k \equiv b^k \ [n], (k \in \N)$ 
\end{enumerate}
\end{proposition}
\end{graybox}

\begin{proof}
On revient à la définition de congruence
\begin{enumerate}
\item $a - a = 0 = 0 \times n \implies a \equiv a \ [n]$
\item $a \equiv b \ [n] \iff a - b = kn, (k \in \Z) \iff b - a = -kn, (k \in \Z) \implies b \equiv a \ [n]$
\item $a \equiv b \ [n] \iff a - b = kn, \ (k \in \Z)$ puis $b \equiv c \ [n] \implies b = c + k'n,\ (k' \in \Z)$ On a donc
\begin{align*}
a - (c + k'n) = kn &\iff a - c - k'n = kn \\
				   &\iff a - c = (k + k')n 
\end{align*}
En posant $(k + k') = K$, $K \in \Z$ par stabilité, ainsi on retrouve 
\begin{align*}
a - c = Kn \iff a \equiv c \ [n]
\end{align*}  
\item 
$a \equiv b \ [n],\ c \equiv d \ [n]$ 
D'après la définition de congruence, on a :
\begin{align*}
\exists k \in \Z,\ a - b &= kn & \exists k' \in \Z,\ c - d &= k'n \\
a &= b + kn & c &= d + k'n
\end{align*}
En faisant la somme des deux égalités on obtient:
\begin{align*}
a + c &= b + d + kn + k'n \\
a + c &= b + d + (k + k')n
\end{align*}
En posant $K = k + k'$, $K \in \Z$ par stabilité on obtient :
\begin{align*}
a + c &= b + d + Kn \iff a + c \equiv b + d \ [n]
\end{align*}
\item 
$a \equiv b \ [n],\ c \equiv d \ [n]$ D'après la définition de congruence on a :
\begin{align*}
\exists k \in \Z,\ a - b &= kn & \exists k' \in \Z,\ c - d &= k'n \\
a &= b + kn & c &= d + k'n
\end{align*}
En faisant le produit des deux égalités on obtient :
\begin{align*}
ac &= (b + kn)(d + k'n) \\
ac &= bd + bk'n + dkn + kk'n^2 \\
ac &= bd + (bk' + dk +kkn)n
\end{align*}
En posant $K = bk' + dk + kk'n$, $K \in \Z$ par stabilité on obtient :
\begin{align*}
ac = bd + Kn \iff ac - bd = Kn \iff ac \equiv bd \ [n]
\end{align*}
\item On procède par récurrence :
\\
Supposons la propriété
\begin{align*}
P_n \colon ``a^p \equiv b^p \ [n], p \in \N"
\end{align*}
\textbf{Initialisation :} Pour $p = 0$ on a :
$a^0 = b^0 = 1$
\begin{align*}
a^0 \equiv b^0 \ [n] \iff 1 \equiv 1 \ [n]
\end{align*}
La propriété $P_0$ est vraie.
\\
\textbf{Hérédité :} Supposons pour un $p > 0$ qu'on ait la propriété 
\begin{align*}
P_n \colon ``a^p \equiv b^p \ [n]"
\end{align*}
Montrons que $P_{n + 1}$ est vraie.\\
D'après l'hypothèse de récurrence :
\begin{align*}
a^p \equiv b^p \ [n]
\end{align*}
Sachant que $a \equiv b \ [n]$, par produit de congruences on obtient :
\begin{align*}
a^p a &\equiv b^p b \ [n] \\
P_{n+1} \colon ``a^{p+1} &\equiv b^{p+1} \ [n]"
\end{align*}
On a montré que $P_0$ est vraie puis que si $P_n$ est vraie alors $P_{n+1}$ est vraie. Ceci achève la récurrence et la propriété est vérifiée.
\end{enumerate}
\end{proof}

\begin{exemple}
\begin{align*}
8^{5000} - 6^{4787} \text{ modulo } 7
\end{align*}
On a : 
$
\begin{cases}
8 \equiv 1 \ [7] \\
6 \equiv -1 \ [7]
\end{cases}
\implies 
\begin{cases}
8^{5000} \equiv 1 \ [7] \\
6^{4787} \equiv -1 \ [7]
\end{cases}
\implies 
\begin{cases}
8^{5000} - 6^{4787} \equiv 2 \ [7]
\end{cases}
$
\end{exemple}

\begin{exemple}
Trouver les $x \in \Z$ tels que 
\begin{align*}
3x \equiv 5 \ [7]
\end{align*}
On a une solution particulière $x_0 = 4$
\\
On a ensuite
\begin{align*}
3x &\equiv 5 \ [7] \\
6x &\equiv 10 \ [7] \\
6x &\equiv 3 \ [7] \\
6x &\equiv -x_0 \ [7] \\
\end{align*}
\begin{align*}
-x_0 \equiv 3 \ [7] \iff x_0 \equiv -3 \ [7] \equiv 4 \ [7]
\end{align*}
On a ensuite 
\begin{align*}
3x &\equiv 5 \ [7] \\
3x_0 &\equiv 5 \ [7] \iff 3 \times 4 \equiv 5 \ [7]
\end{align*}
On a donc :
\begin{align*}
3(x - x_0) &\equiv 0 \ [7] 
\end{align*}
\begin{align*}
3(x - 4) \equiv 0 \ [7] &\iff 3(x - 4) = 7k,\ (k \in \Z) \\
						&\iff 7 \mid 3(x - 4) \\
\end{align*}
Par le Lemme de Gauss :
\begin{align*}
7 \nmid 3 &\implies 7 \mid x - 4 \\
		  &\implies x - 4 = 7k,\ (k \in \Z) \\
		  &\implies x = 7k + 4,\ (k \in \Z)
\end{align*}
\end{exemple}
On doit être capable de résoudre deux types d'équations
\begin{itemize}
    \item Equation diophantienne : $a, b \in \N^*, \mathrm{pgcd}(a, b) = d$
        \begin{align*}
            (E) : ax + by = d,\ (x, y) \in \Z^2
        \end{align*}
    \item Equation à congruences
\end{itemize}

\begin{remarque}[Méthode pour les équations diophantiennes]~
    \begin{enumerate}
        \item On trouve une solution particulière avec l'identité de Bézout.
        \item On résout ensuite l'équation homogène à l'aide du Lemme de Gauss.
        \item On regroupe les deux équations et on résout le système.
    \end{enumerate}
\end{remarque}

\begin{remarque}[Méthode pour les équations à congruences]~
    \begin{enumerate}
        \item On trouve une solution particulière avec l'identité de Bézout en réécrivant la congruence sous une forme d'égalité.
        \item On résout l'équation homogène à l'aide du Lemme de Gauss.
        \item On regroupe les deux équations et on résout le système.
    \end{enumerate}
\end{remarque}

\begin{remarque}
Bien vérifier que l'équation est solvable en vérifiant si le membre de droite est un multiple du pgcd.
\end{remarque}

\begin{remarque}
$a_i \in \Z, m_i \in \N^*, (i = 1, 2)$ tel que
\begin{itemize}
    \item $\mathrm{pgcd}(m_1, m_2) = 1$
    \item $m_i > 1$, $\forall i$
\end{itemize}
\begin{align*}
    (E) \colon x \equiv a_i \ [m_i],\ (i=1,2)
\end{align*}
\begin{itemize}
    \item Identité de Bézout : $\exists (u_1, u_2) \in \Z^2$ tel que $m_1u_1 + m_2u_2 = 1$
\begin{align*}
S_0 = a_1m_2u_2 + a_2m_1u_1
\end{align*}
\begin{align*}
    S_0 &\equiv a_1 m_2 u_2 \ [m_1] \\
        &\equiv a_1(1 - m_1 u_1) \ [m_1] \\
        &\equiv a_1 \ [m_1] \\
        &\equiv a_2 m_1 u_1 \ [m_2] \\
        &\equiv a_2 (1 - m_2 u_2) \ [m_2] \\
        &\equiv a_2 \ [m_2]
\end{align*}
$x = S_0$ est une solution particulière de $(E)$
\end{itemize}
On a :
\begin{align*}
    \begin{cases}
x &\equiv a_i \ [m_i] \\
      S_0 &\equiv a_i \ [m_i]
    \end{cases}
    \implies 
    x - S_0 \equiv 0 \implies \exists k \in \Z, x - S_0 = m_1 m_2 k
\end{align*} 
d'où l'ensemble des solution de $(E)$ est 
\begin{align*}
S_0 + m_1 m_2
\end{align*}
la division euclidienne, $\exists ! (q, k_0) \in \Z^2$ tel que
\begin{enumerate}
    \item $S_0 = q m_1 m_2 + k_0$ 
    \item $0 \leq k_0 < m_1 m_2$
\end{enumerate}
\begin{align*}
    S_0 + m_1 m_2 \Z &= k_0 + m_1 m_2 q + m_1 m_2 \Z \\
                     &= k_0 + m_1 m_2 (q + \Z) \\ 
                     &= k_0 + m_1 m_2 \Z
\end{align*}
\end{remarque}

\begin{graybox}
    \begin{theoreme}[Théorème des restes chinois]
        Soient $m_1, m_2$ deux entiers naturels tels que 
        \begin{itemize}
            \item $m_1 > 1$
            \item $\mathrm{pgcd}(m_1, m_2) = 1$
        \end{itemize}
        Soient $a_i \in \Z, (i = 1, 2)$
        \\
        Notons l'ensemble des solutions des systèmes d'équations par $\mathcal{S}$.
        \begin{align*}
            x &\equiv a_1 \ [m_1] & x &\equiv a_2 \ [m_2]
        \end{align*}
    \end{theoreme}
    Alors 
    \begin{align*}
        \exists k_0 \in \Z \text{ tel que } 
    \end{align*}
    \begin{itemize}
        \item $\mathcal{S} = k_0 + m_1 m_2 \Z$
        \item $0 \leq k_0 < m_1 m_2$
    \end{itemize}
\end{graybox}
